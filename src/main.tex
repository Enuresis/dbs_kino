%! Author = jdubec
%! Date = 16/02/2026

\documentclass[11pt, a4paper]{article}

% ─── Packages ────────────────────────────────────────────────────────────────
\usepackage[utf8]{inputenc}
\usepackage[T1]{fontenc}
\usepackage[margin=2.5cm]{geometry}
\usepackage{graphicx}
\usepackage{hyperref}
\usepackage{booktabs}       % krajsie tabulky (\toprule, \midrule, \bottomrule)
\usepackage{float}          % [H] umiestnenie pre obrazky a tabulky
\usepackage{xcolor}         % farby (pouzite v listings)
\usepackage{listings}       % vypis kodu
\hypersetup{colorlinks=true, linkcolor=black, citecolor=black, urlcolor=blue!70!black}

% ─── Code listing style ─────────────────────────────────────────────────────
\lstset{
    basicstyle=\ttfamily\small,
    breaklines=true,
    frame=single,
    numbers=left,
    numberstyle=\tiny\color{gray},
    xleftmargin=2em,
    framexleftmargin=1.5em
}

% ─── Bibliography ────────────────────────────────────────────────────────────
\usepackage[style=numeric, backend=bibtex]{biblatex}
\addbibresource{references.bib}

% ─── Language────────────────────────────────────────────
\input{lang/sk}

% ─── Detaily studenta ────────────────────────────────────
\newcommand{\StudentName}{Peter Mezei, David Blanko}
\newcommand{\StudyGroup}{Utorok 11:00}
\newcommand{\AssignmentNumber}{1}
\newcommand{\AssignmentTitle}{Návrh databázového modelu - Správa kina}
\newcommand{\AcademicYear}{2025/2026}

\begin{document}

% ─── Titulna stranka ─────────────────────────────────────────────────────────
\begin{titlepage}
    \centering

    \includegraphics[width=0.3\textwidth]{../assets/logo}

    \vspace{0.3cm}
    {\large \LabelUniversity}\\[2pt]
    {\large \LabelFaculty}

    \vspace{2.5cm}
    {\Huge\bfseries \LabelProtocol}

    \vspace{0.6cm}
    {\LARGE \LabelCode{} --- \LabelCourse}

    \vspace{0.4cm}
    {\Large \LabelAssignment{} \AssignmentNumber: \AssignmentTitle}

    \vfill

    \begin{tabular}{r l}
        \textbf{\LabelStudent:} & \StudentName  \\[4pt]
        \textbf{\LabelGroup:}   & \StudyGroup   \\[4pt]
        \textbf{\LabelYear:}    & \AcademicYear \\[4pt]
        \textbf{\LabelDate:}    & \today        \\
    \end{tabular}

    \vspace{1.5cm}
\end{titlepage}

% ─── Obsah ───────────────────────────────────────────────────────────────────
\renewcommand{\contentsname}{\LabelContents}
\tableofcontents
\newpage

\section{Špecifikácia}
\label{sec:specifikacia}

\subsection{Aký problém systém rieši a pre koho je určený?}
Systém rieši centralizovanú správu kina, konkrétne evidenciu filmov, plánovanie premietaní v sálach, predaj lístkov na konkrétne sedadlá a manažment úloh zamestnancov. Nahrádza manuálnu evidenciu, čím eliminuje duplicitu predaných miest a kolízie v harmonograme sál.

Systém je určený primárne pre manažéra kina (administrácia) a radových zamestnancov (predaj, údržba). Pridanou hodnotou je automatizácia kontroly kapacít a zefektívnenie prideľovania práce.

\subsection{Kto sú aktéri systému?}
\begin{itemize}
    \item \textbf{Manažér:} Má plné práva. Spravuje číselníky (filmy, sály), vytvára harmonogram premietaní a prideľuje pracovné úlohy zamestnancom.
    \item \textbf{Zamestnanec (Pokladník / Uvádzač):} Realizuje predaj lístkov a prehliada si zoznam svojich pridelených úloh (napr. upratovanie sály, kontrola lístkov).
    \item \textbf{Zákazník:} Neregistrovaný používateľ, ktorý si prezerá program a kupuje lístky.
\end{itemize}

\subsection{Aké sú kľúčové entity?}
\begin{itemize}
    \item \textbf{Film:} Základná entita nesúca informácie o diele (názov, dĺžka, žáner, vekové obmedzenie).
    \item \textbf{Kinosála:} Reprezentuje fyzickú miestnosť. Má definovaný typ (2D, 3D) a väzbu na sedadlá.
    \item \textbf{Sedadlo:} Konkrétne miesto v sále (rad, číslo).
    \item \textbf{Premietanie:} Spája Film a Kinosálu v konkrétnom čase.
    \item \textbf{Lístok:} Záznam o predaji. Viaže sa na jedno Premietanie a jedno konkrétne Sedadlo.
    \item \textbf{Zamestnanec:} Osoba v pracovnom pomere. Obsahuje atribúty ako typ úväzku a platnosť zdravotného preukazu (pre prácu s občerstvením).
    \item \textbf{Úloha:} Konkrétna pracovná činnosť pridelená zamestnancovi v určitom čase (napr. upratovanie sály po premietaní).
\end{itemize}

\subsection{Biznis pravidlá a integritné obmedzenia}

\begin{enumerate}
    \item \textbf{Unikátnosť lístka:} Dva platné lístky nemôžu odkazovať na rovnaké sedadlo v rámci toho istého premietania.
    \item \textbf{Časová exkluzivita sály:} V jednej kinosále nemôžu prebiehať dve premietania súčasne. Začiatok nového premietania musí byť neskôr ako koniec predchádzajúceho (vrátane času na údržbu).
    \item \textbf{Kapacita sály:} Počet lístkov na premietanie nesmie prekročiť počet existujúcich sedadiel prislúchajúcich danej sále.
    \item \textbf{Oprávnenie zamestnanca:} Zamestnanec môže vykonávať úlohy s jedlom len vtedy, ak má v systéme evidovaný platný zdravotný preukaz.
    \item \textbf{Dostupnosť zamestnanca:} Jeden zamestnanec nemôže mať pridelené dve časovo sa prekrývajúce úlohy.
    \item \textbf{Konzistencia sedadiel:} Lístok je možné vytvoriť len na sedadlo, ktoré fyzicky patrí do kinosály, v ktorej sa koná dané premietanie.
\end{enumerate}

\newpage
\section{Definovanie používateľských scenárov}
\label{sec:scenare}

\subsection*{Scenár 1: Naplánovanie harmonogramu premietania}
\begin{itemize}
    \item \textbf{Názov a stručný popis:} Vytvorenie inštancie entity \textit{Premietanie}. Manažér priraďuje konkrétny \textit{Film} do zvolenej \textit{Kinosály} na určený časový interval. Scenár primárne rieši validáciu časovej exkluzivity sály.
    \item \textbf{Hlavný tok:}
    \begin{enumerate}
        \item Manažér inicializuje proces tvorby nového premietania a vyberie záznam z číselníka filmov.
        \item Systém načíta atribút \textit{dĺžka} z vybraného filmu.
        \item Manažér definuje \textit{čas\_začiatku} a vyberie cieľovú kinosálu.
        \item Systém algoritmicky vypočíta \textit{čas\_konca} (čas začiatku + dĺžka filmu + definovaná konštanta pre údržbu/upratovanie sály).
        \item Systém vykoná dopyt na existujúce premietania pre danú kinosálu a overí, či nedochádza k prieniku časových intervalov.
        \item Transakcia je potvrdená (commit) a záznam je perzistentne uložený.
    \end{enumerate}
    \item \textbf{Dátova vrstva}
    \begin{itemize}
        \item \textit{Čítanie:} \texttt{SELECT} nad tabuľkou \texttt{Film} pre získanie dĺžky. Následne \texttt{SELECT} nad tabuľkou \texttt{Premietanie} s podmienkou pre konkrétny cudzí kľúč (\texttt{FK\_Kinosala}) a vyhodnotením prekrývania časových pečiatok (\texttt{DATETIME}).
        \item \textit{Zápis:} \texttt{INSERT} operácia do väzobnej/asociačnej tabuľky \texttt{Premietanie}, čím vznikajú cudzie kľúče na \texttt{Film} a \texttt{Kinosálu}.
        \item \textit{Kontrola pravidiel:} Uplatňuje sa biznis pravidlo č. 2.
    \end{itemize}
    \item \textbf{Alternatívne toky:} Ak databázový dopyt vráti existujúci záznam v kolidujúcom čase, systém vyvolá výnimku, zruší zápis a vyzve manažéra na úpravu vstupných parametrov (iná sála alebo iný čas).
    \item \textbf{UML diagram:} \textbf{Este treba Diagram aktivít}.
\end{itemize}

\subsection*{Scenár 2: Transakčný predaj lístkov s alokáciou sedadiel}
\begin{itemize}
    \item \textbf{Názov a stručný popis:} Zamestnanec (pokladník) realizuje predaj lístkov. Scenár vyžaduje načítanie priestorovej mapy sály, kontrolu dostupnosti a vytvorenie záznamov rešpektujúcich unikátnosť sedadla pre dané premietanie.
    \item \textbf{Hlavný tok:}
    \begin{enumerate}
        \item Pokladník vyberie inštanciu aktuálneho premietania.
        \item Systém načíta topológiu sály a vyhodnotí obsadenosť jednotlivých sedadiel.
        \item Pokladník označí množinu voľných sedadiel (napr. 2 lístky) a inicializuje platbu.
        \item Systém bezprostredne pred zápisom overí aktuálnosť stavu.
        \item Systém vygeneruje záznamy pre lístky a označí ich ako predané.
    \end{enumerate}
    \item \textbf{Dátova vrstva}
    \begin{itemize}
        \item \textit{Čítanie:} Vykonáva sa relačné spojenie (\texttt{JOIN}) tabuliek \texttt{Kinosala} a \texttt{Sedadlo}. Na zistenie voľných miest sa použije \texttt{LEFT JOIN} na tabuľku \texttt{Listok} filtrovaný podľa \texttt{FK\_Premietanie}. Záznamy s hodnotou \texttt{NULL} v spojenej tabuľke lístkov reprezentujú voľné sedadlá.
        \item \textit{Zápis:} Vykoná sa viacnásobný \texttt{INSERT} do tabuľky \texttt{Listok}. Vznikajú väzby (\texttt{FK}) na \texttt{Premietanie} a konkrétne \texttt{Sedadlo}.
        \item \textit{Kontrola pravidiel:} Uplatňuje sa integritné obmedzenie \texttt{UNIQUE(FK\_Premietanie, FK\_Sedadlo)} na databázovej úrovni, čo garantuje pravidlo č. 1 (Unikátnosť lístka). Zabezpečuje sa aj pravidlo č. 6 (Konzistencia sedadiel).
    \end{itemize}
    \item \textbf{Alternatívne toky:} Ak v čase medzi načítaním mapy sály a potvrdením transakcie iný používateľ obsadí vybrané sedadlo (porušenie \texttt{UNIQUE} constraintu), databáza odmietne \texttt{INSERT}. Transakcia sa roll-backne a používateľovi sa aktualizuje mapa s chybovým hlásením.
    \item \textbf{UML diagram:} \textbf{Este treba Sekvenčný diagram}.
\end{itemize}

\subsection*{Scenár 3: Alokácia personálnych kapacít na prevádzkové úlohy}
\begin{itemize}
    \item \textbf{Názov a stručný popis:} Manažér priraďuje konkrétnu inštanciu entity \textit{Úloha} (napr. predaj v bufete) zamestnancovi. Systém filtruje zamestnancov na základe ich kvalifikačných obmedzení (zdravotný preukaz) a časovej disponibility.
    \item \textbf{Hlavný tok:}
    \begin{enumerate}
        \item Manažér definuje typ úlohy a vyžadovaný časový interval.
        \item Systém načíta a aplikuje filtre na záznamy v tabuľke zamestnancov na základe typu úlohy.
        \item Zobrazí sa podmnožina zamestnancov, ktorí spĺňajú podmienky.
        \item Manažér vyberie zamestnanca a potvrdí alokáciu.
        \item Systém verifikuje, či zamestnanec nemá v danom čase priradenú inú úlohu.
        \item Systém vytvorí asociáciu medzi zamestnancom a úlohou.
    \end{enumerate}
    \item \textbf{Dátova vrstva}
    \begin{itemize}
        \item \textit{Čítanie:} Filtrovací \texttt{SELECT} nad entitou \texttt{Zamestnanec} s podmienkou v klauzule \texttt{WHERE}, ktorá overuje, či \texttt{platnost\_zdravotneho\_preukazu >= CURRENT\_DATE} (ak ide o prácu s jedlom). Následný dopyt nad tabuľkou \texttt{Uloha} overuje prienik časových intervalov pre dané \texttt{id\_zamestnanca}.
        \item \textit{Zápis:} \texttt{INSERT} nového záznamu do tabuľky \texttt{Uloha} s príslušným \texttt{FK\_Zamestnanec}.
        \item \textit{Kontrola pravidiel:} Uplatnenie biznis pravidiel č. 4 (Oprávnenie) a č. 5 (Dostupnosť zamestnanca).
    \end{itemize}
    \item \textbf{Alternatívne toky:} V prípade, že pre daný typ úlohy v požadovanom čase neexistuje žiadny dostupný a kvalifikovaný zamestnanec, systém vráti prázdny \textit{result set} a informuje manažéra o podstave personálu.
    \item \textbf{UML diagram:} \textbf{Este treba Diagram aktivít}.
\end{itemize}

\subsection*{Scenár 4: Generovanie agregovaných štatistík predajnosti}
\begin{itemize}
    \item \textbf{Názov a stručný popis:} Generovanie manažérskeho reportu. Ide o analytický dopyt, ktorý agreguje transakčné dáta za určité časové obdobie a zoskupuje ich podľa jednotlivých filmových titulov.
    \item \textbf{Hlavný tok:}
    \begin{enumerate}
        \item Manažér si vyžiada štatistický prehľad a definuje časový parameter (napr. konkrétny deň).
        \item Systém spustí analytický dopyt naprieč relačnou štruktúrou dát.
        \item Databázový stroj prepojí filmy, ich premietania a prislúchajúce predané lístky.
        \item Systém vypočíta metriky: celkový počet predaných lístkov a percentuálnu obsadenosť.
        \item Vrátená množina dát (dataset) je formátovaná do prehľadnej tabuľky pre manažéra.
    \end{enumerate}
    \item \textbf{Dátova vrstva}
    \begin{itemize}
        \item \textit{Čítanie a Agregácia:} Nevykonáva sa žiadny zápis. Dominantný je komplexný \texttt{SELECT} spájajúci tabuľky \texttt{Film}, \texttt{Premietanie} a \texttt{Listok} cez príslušné cudzie kľúče.
        \item Využívajú sa agregačné funkcie: \texttt{COUNT(id\_listka)} pre zisťovanie návštevnosti.
        \item Dáta sú štruktúrované pomocou \texttt{GROUP BY id\_filmu}, pričom sa aplikuje filter \texttt{WHERE} na ohraničenie časového intervalu (\texttt{cas\_zaciatku}).
    \end{itemize}
    \item \textbf{Alternatívne toky:} Ak sa vo zvolenom dni nerealizovalo žiadne premietanie (alebo sa nepredali žiadne lístky), spojený dopyt vráti prázdnu množinu a používateľské rozhranie informuje aktéra o nulovej návštevnosti.
    \item \textbf{UML diagram:} \textbf{Este treba Sekvenčný diagram}.
\end{itemize}

\newpage
\section{Návrh relačného modelu}
\label{sec:relacny_model}

\textit{Tu navrhy}

\newpage
\section{Zhodnotenie}
\label{sec:zhodnotenie}

\textit{zhodnotenie}

\printbibliography

\end{document}
