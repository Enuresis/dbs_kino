%! Author = jdubec
%! Date = 16/02/2026

\documentclass[11pt, a4paper]{article}

% ─── Packages ────────────────────────────────────────────────────────────────
\usepackage[utf8]{inputenc}
\usepackage[T1]{fontenc}
\usepackage[margin=2.5cm]{geometry}
\usepackage{graphicx}
\usepackage{hyperref}
\usepackage{booktabs}       % krajsie tabulky (\toprule, \midrule, \bottomrule)
\usepackage{float}          % [H] umiestnenie pre obrazky a tabulky
\usepackage{xcolor}         % farby (pouzite v listings)
\usepackage{listings}       % vypis kodu
\hypersetup{colorlinks=true, linkcolor=black, citecolor=black, urlcolor=blue!70!black}

% ─── Code listing style ─────────────────────────────────────────────────────
\lstset{
    basicstyle=\ttfamily\small,
    breaklines=true,
    frame=single,
    numbers=left,
    numberstyle=\tiny\color{gray},
    xleftmargin=2em,
    framexleftmargin=1.5em
}

% ─── Bibliography ────────────────────────────────────────────────────────────
\usepackage[style=numeric, backend=bibtex]{biblatex}
\addbibresource{references.bib}

% ─── Language────────────────────────────────────────────
\input{lang/sk}

% ─── Detaily studenta ────────────────────────────────────
\newcommand{\StudentName}{Peter Mezei, David Blanko}
\newcommand{\StudyGroup}{Utorok 11:00}
\newcommand{\AssignmentNumber}{1}
\newcommand{\AssignmentTitle}{Návrh databázového modelu - Správa kina}
\newcommand{\AcademicYear}{2025/2026}

\begin{document}

% ─── Titulna stranka ─────────────────────────────────────────────────────────
\begin{titlepage}
    \centering

    \includegraphics[width=0.3\textwidth]{../assets/logo}

    \vspace{0.3cm}
    {\large \LabelUniversity}\\[2pt]
    {\large \LabelFaculty}

    \vspace{2.5cm}
    {\Huge\bfseries \LabelProtocol}

    \vspace{0.6cm}
    {\LARGE \LabelCode{} --- \LabelCourse}

    \vspace{0.4cm}
    {\Large \LabelAssignment{} \AssignmentNumber: \AssignmentTitle}

    \vfill

    \begin{tabular}{r l}
        \textbf{\LabelStudent:} & \StudentName  \\[4pt]
        \textbf{\LabelGroup:}   & \StudyGroup   \\[4pt]
        \textbf{\LabelYear:}    & \AcademicYear \\[4pt]
        \textbf{\LabelDate:}    & \today        \\
    \end{tabular}

    \vspace{1.5cm}
\end{titlepage}

% ─── Obsah ───────────────────────────────────────────────────────────────────
\renewcommand{\contentsname}{\LabelContents}
\tableofcontents
\newpage

\section{Špecifikácia}
\label{sec:specifikacia}

\subsection{Aký problém systém rieši a pre koho je určený?}
Systém rieši centralizovanú správu kina, konkrétne evidenciu filmov, plánovanie premietaní v sálach, predaj lístkov na konkrétne sedadlá a manažment úloh zamestnancov. Nahrádza manuálnu evidenciu, čím eliminuje duplicitu predaných miest a kolízie v harmonograme sál.

Systém je určený primárne pre manažéra kina (administrácia) a radových zamestnancov (predaj, údržba). Pridanou hodnotou je automatizácia kontroly kapacít a zefektívnenie prideľovania práce.

\subsection{Kto sú aktéri systému?}
\begin{itemize}
    \item \textbf{Manažér:} Má plné práva. Spravuje číselníky (filmy, sály), vytvára harmonogram premietaní a prideľuje pracovné úlohy zamestnancom.
    \item \textbf{Zamestnanec (Pokladník / Uvádzač):} Realizuje predaj lístkov a prehliada si zoznam svojich pridelených úloh (napr. upratovanie sály, kontrola lístkov).
    \item \textbf{Zákazník:} Neregistrovaný používateľ, ktorý si prezerá program a kupuje lístky.
\end{itemize}

\subsection{Aké sú kľúčové entity?}
\begin{itemize}
    \item \textbf{Film:} Základná entita nesúca informácie o diele (názov, dĺžka, žáner, vekové obmedzenie).
    \item \textbf{Kinosála:} Reprezentuje fyzickú miestnosť. Má definovaný typ (2D, 3D) a väzbu na sedadlá.
    \item \textbf{Sedadlo:} Konkrétne miesto v sále (rad, číslo).
    \item \textbf{Premietanie:} Spája Film a Kinosálu v konkrétnom čase.
    \item \textbf{Lístok:} Záznam o predaji. Viaže sa na jedno Premietanie a jedno konkrétne Sedadlo.
    \item \textbf{Zamestnanec:} Osoba v pracovnom pomere. Obsahuje atribúty ako typ úväzku a platnosť zdravotného preukazu (pre prácu s občerstvením).
    \item \textbf{Úloha:} Konkrétna pracovná činnosť pridelená zamestnancovi v určitom čase (napr. upratovanie sály po premietaní).
\end{itemize}

\subsection{Biznis pravidlá a integritné obmedzenia}

\begin{enumerate}
    \item \textbf{Unikátnosť lístka:} Dva platné lístky nemôžu odkazovať na rovnaké sedadlo v rámci toho istého premietania.
    \item \textbf{Časová exkluzivita sály:} V jednej kinosále nemôžu prebiehať dve premietania súčasne. Začiatok nového premietania musí byť neskôr ako koniec predchádzajúceho (vrátane času na údržbu).
    \item \textbf{Kapacita sály:} Počet lístkov na premietanie nesmie prekročiť počet existujúcich sedadiel prislúchajúcich danej sále.
    \item \textbf{Oprávnenie zamestnanca:} Zamestnanec môže vykonávať úlohy s jedlom len vtedy, ak má v systéme evidovaný platný zdravotný preukaz.
    \item \textbf{Dostupnosť zamestnanca:} Jeden zamestnanec nemôže mať pridelené dve časovo sa prekrývajúce úlohy.
    \item \textbf{Konzistencia sedadiel:} Lístok je možné vytvoriť len na sedadlo, ktoré fyzicky patrí do kinosály, v ktorej sa koná dané premietanie.
\end{enumerate}

\newpage
\section{Definovanie používateľských scenárov}
\label{sec:scenare}

\textit{potom doplnit scenare}

\newpage
\section{Návrh relačného modelu}
\label{sec:relacny_model}

\textit{Tu navrhy}

\newpage
\section{Zhodnotenie}
\label{sec:zhodnotenie}

\textit{zhodnotenie}

% ─── Bibliography ────────────────────────────────────────────────────────────
\printbibliography

\end{document}
